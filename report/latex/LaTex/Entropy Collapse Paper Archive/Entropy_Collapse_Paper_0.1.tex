
\documentclass[12pt]{article}
\usepackage{amsmath, amssymb, graphicx, hyperref, geometry}
\geometry{margin=1in}
\title{Entropy Collapse: A Unified Framework Linking Entropy, Gravity, and Fundamental Particle Formation}
\author{Thought Experiment Research Team}
\date{\today}

\begin{document}

\maketitle

\begin{abstract}
We present a unified framework proposing that entropy is the fundamental organizing principle of physical reality, and that at extreme compression thresholds—such as inside black holes—entropy collapses from a 3D volumetric distribution to a 2D surface-projected state. This entropy collapse governs the stabilization of mass-energy and the emergence of fundamental particles. We explore a boundary condition known as the Bruno Constant, the threshold where this collapse occurs. Our simulations, figure mappings, and chronological script records establish this theory’s internal consistency and reveal new testable pathways bridging general relativity and quantum mechanics.
\end{abstract}

\tableofcontents

\section{Introduction}
\label{sec:intro}

% Intro content will go here...

\section{Foundational Assumptions}
\label{sec:foundations}

% Definitions, entropy-first logic, Bruno Constant, etc.

\section{Entropy Collapse Mechanism}
\label{sec:mechanism}

% Equations, collapse model, 3D to 2D logic

\section{Figure Analysis and Simulation Results}
\label{sec:figures}

% Figures matched with equation waves

\section{Dimensional Stabilization and Quantum Convergence}
\label{sec:dimensional}

% 0K stabilization, superconductivity analogies, holographic overlay

\section{Discussion and Challenges}
\label{sec:discussion}

% Weaknesses, open questions, empirical tests

\section{Conclusion and Next Steps}
\label{sec:conclusion}

% Summary and future direction

\appendix

\section{Appendix A: Timeline of Script Development}
\label{appendix:scripts}

\section{Appendix B: Figure Development Phases}
\label{appendix:figures}

\end{document}
